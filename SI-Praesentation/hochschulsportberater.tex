% $Header: /Users/joseph/Documents/LaTeX/beamer/solutions/generic-talks/generic-ornate-15min-45min.de.tex,v 90e850259b8b 2007/01/28 20:48:30 tantau $

\documentclass{beamer}

% Diese Datei enthält eine Lösungsvorlage für:


% - Vorträge über ein beliebiges Thema.
% - Vortragslänge zwischen 15 und 45 Minuten. 
% - Aussehen des Vortrags ist verschnörkelt/dekorativ.



% Copyright 2004 by Till Tantau <tantau@users.sourceforge.net>.
%
% In principle, this file can be redistributed and/or modified under
% the terms of the GNU Public License, version 2.
%
% However, this file is supposed to be a template to be modified
% for your own needs. For this reason, if you use this file as a
% template and not specifically distribute it as part of a another
% package/program, I grant the extra permission to freely copy and
% modify this file as you see fit and even to delete this copyright
% notice. 



\mode<presentation>
{
  %\usetheme{CambridgeUS}
  % oder ...
  \usecolortheme{beaver}
  \useoutertheme[subsection=false]{miniframes}
  
\setbeamertemplate{footline}
{%
  \leavevmode%
  \hbox{%
		\begin{beamercolorbox}[wd=.333333\paperwidth,ht=2.25ex,dp=1ex,left]{subsection in head/foot}%
    		\hspace*{2ex}\usebeamerfont{subsection in head/foot}\insertsubsection
  		\end{beamercolorbox}%
  		\begin{beamercolorbox}[wd=.333333\paperwidth,ht=2.25ex,dp=1ex,center]{title in head/foot}%
    		\usebeamerfont{title in head/foot}\insertshorttitle
  		\end{beamercolorbox}%
  		\begin{beamercolorbox}[wd=.333333\paperwidth,ht=2.25ex,dp=1ex,right]{date in head/foot}%
    		\usebeamerfont{date in head/foot}\insertshortdate{}\hspace*{2em}
   			\insertframenumber{}\hspace*{2ex} 
  		\end{beamercolorbox}
  }%
  \vskip0pt%
  
	
}
  
  
  \setbeamercovered{transparent}
  % oder auch nicht
  
	\setbeamercolor{itemize item}{fg=darkred}
	\setbeamercolor{itemize subitem}{fg=darkred}
	\setbeamercolor{enumerate item}{fg=darkred}
	
	\setbeamertemplate{navigation symbols}{}%remove navigation symbols
}


\usepackage[german]{babel}
% oder was auch immer

\usepackage[utf8]{inputenc}
% oder was auch immer

\usepackage{times}
\usepackage[T1]{fontenc}
% Oder was auch immer. Zu beachten ist, das Font und Encoding passen
% müssen. Falls T1 nicht funktioniert, kann man versuchen, die Zeile
% mit fontenc zu löschen.


\title[Hochschulsportberater] % (optional, nur bei langen Titeln nötig)
{Semantische Interoperabilität}

\subtitle
{Hochschulsportberater} % (optional)

\author{Gilles Baatz, Jan Jochum, Tobias Kalmes, Frantz Tenguemne, Michael Wolf}
% - Der \inst{?} Befehl sollte nur verwendet werden, wenn die Autoren
%   unterschiedlichen Instituten angehören.

\institute[htw saar] % (optional, aber oft nötig)
{
  HTW des Saarlandes
}  
  


% - Der \inst{?} Befehl sollte nur verwendet werden, wenn die Autoren
%   unterschiedlichen Instituten angehören.
% - Keep it simple, niemand interessiert sich für die genau Adresse.

\date[Semantische Interoperabilität] % (optional)
{\today}


\subject{Master Praktische Informatik}
% Dies wird lediglich in den PDF Informationskatalog einfügt. Kann gut
% weggelassen werden.


% Falls eine Logodatei namens "university-logo-filename.xxx" vorhanden
% ist, wobei xxx ein von latex bzw. pdflatex lesbares Graphikformat
% ist, so kann man wie folgt ein Logo einfügen:

% \pgfdeclareimage[height=0.5cm]{university-logo}{university-logo-filename}
% \logo{\pgfuseimage{university-logo}}



% Folgendes sollte gelöscht werden, wenn man nicht am Anfang jedes
% Unterabschnitts die Gliederung nochmal sehen möchte.
\AtBeginSection[]
{
  \begin{frame}<beamer>{Gliederung}
    \tableofcontents[currentsection]
  \end{frame}
}


% Falls Aufzählungen immer schrittweise gezeigt werden sollen, kann
% folgendes Kommando benutzt werden:

%\beamerdefaultoverlayspecification{<+->}



\begin{document}

\begin{frame}
  \titlepage
\end{frame}



% Da dies ein Vorlage für beliebige Vorträge ist, lassen sich kaum
% allgemeine Regeln zur Strukturierung angeben. Da die Vorlage für
% einen Vortrag zwischen 15 und 45 Minuten gedacht ist, fährt man aber
% mit folgenden Regeln oft gut.  

% - Es sollte genau zwei oder drei Abschnitte geben (neben der
%   Zusammenfassung). 
% - *Höchstens* drei Unterabschnitte pro Abschnitt.
% - Pro Rahmen sollte man zwischen 30s und 2min reden. Es sollte also
%   15 bis 30 Rahmen geben.



\section{Einleitung}

\subsection[Kurzversion des ersten Unterabschnittstitels]
{Erster Unterabschnittstitel}

\begin{frame}{Überschriften müssen informativ sein.\\
    Korrekte Groß-/Kleinschreibung beachten.}{Untertitel sind optional.}
  % - Eine Überschrift fasst einen Rahmen verständlich zusammen. Man
  %   muss sie verstehen können, selbst wenn man nicht den Rest des
  %   Rahmens versteht.

  \begin{itemize}
  \item
    Viel \texttt{itemize} benutzen.
  \item
    Sehr kurze Sätze oder Satzglieder verwenden.
  \end{itemize}
\end{frame}

\begin{frame}{Überschriften müssen informativ sein.}

  Man kann Overlays erzeugen\dots
  \begin{itemize}
  \item mit dem \texttt{pause}-Befehl:
    \begin{itemize}
    \item
      Erster Punkt.
      \pause
    \item    
      Zweiter Punkt.
    \end{itemize}
  \item
    mittels Overlay-Spezifikationen:
    \begin{itemize}
    \item<3->
      Erster Punkt.
    \item<4->
      Zweiter Punkt.
    \end{itemize}
  \item
    mit dem allgemeinen \texttt{uncover}-Befehl:
    \begin{itemize}
      \uncover<5->{\item
        Erster Punkt.}
      \uncover<6->{\item
        Zweiter Punkt.}
    \end{itemize}
  \end{itemize}
\end{frame}


\subsection{Zweiter Unterabschnittstitel}

\begin{frame}{Überschriften müssen informativ sein.}
\end{frame}

\begin{frame}{Überschriften müssen informativ sein.}
\end{frame}

\section*{Zusammenfassung}

\begin{frame}{Zusammenfassung}

  % Die Zusammenfassung sollte sehr kurz sein.
  \begin{itemize}
  \item
    Die \alert{erste Hauptbotschaft} des Vortrags in ein bis zwei Zeilen.
  \item
    Die \alert{zweite Hauptbotschaft} des Vortrags in ein bis zwei Zeilen.
  \item
    Eventuell noch eine \alert{dritte Botschaft}, aber nicht noch mehr.
  \end{itemize}
  
  % Der folgende Ausblick ist optional.
  \vskip0pt plus.5fill
  \begin{itemize}
  \item
    Ausblick
    \begin{itemize}
    \item
      Etwas, was wir noch nicht lösen konnten.
    \item
      Nochwas, das wir noch nicht lösen konnten.
    \end{itemize}
  \end{itemize}
\end{frame}


\end{document}


