\chapter{Verbesserungsmöglichkeiten und Fazit}

\section{Verbesserungsmöglichkeiten}

Da es nicht das Ziel des Projektes war, ein möglichst gut funktionierendes Produkt herzustellen, sondern den Umgang mit Ontologien zu erproben, wurden einige Vereinfachungen vorgenommen. Die daraus resultierenden Verbesserungsmöglichkeiten werden im folgenden beschrieben. 

Es ist zu berücksichtigen, dass jede neue Sportart sowohl in der Ontologie, als auch in der Datenbank, mitsamt allen dazugehörigen Informationen, definiert werden muss. Dies kann sehr arbeitsintensiv werden. Bezüglich der Software bietet sich an, eine entsprechende Erweiterung zu entwickeln, die Sportarten neu anlegen und bearbeiten kann. 

Die Datenbank liegt auf einem Server im Internet. In der Realität wäre eine Stand-Alone-Anwendung denkbar. In diesem Fall stellt sich die Frage, ob eine andere Form der Datenhaltung als eine MySQL-Datenbank zweckmäßiger wäre. Weiterhin sind die Zugangsdaten hardcodiert, deshalb ist auf Datenbankseite auch darauf zu achten, dass entsprechende Sicherheitsvorkehrungen getroffen werden (z. B. kein Schreibzugriff auf die Datenbank).

Des weiteren kann die Suche nach einer passenden Sportart für den Benutzer effizienter gemacht werden. Im Augenblick besteht diese aus vielen einzelnen Queries, die nacheinander ausgeführt werden. Diese könnte man wohl auf zwei einzelne Queries reduzieren, wobei eine für die Ontologie und die andere für die Datenbank bestimmt ist. 

Außerdem gibt es noch Optimierungsmöglichkeiten bei der GUI. 
So ist z.B. die Zeitangabe in einem Unterfenster nicht optimal, da bei Angaben einzelner Stunden am Tag für jede Stunde ein Klick gemacht werden muss. Dies wäre durch zielgerichteteres Eventhandling oder eine Andere Zeiterfassung möglich.

\section{Fazit}

Die Lösung hat zu einem vorzeigbaren Ergebnis geführt, die auch über eine intuitive Bedienbarkeit verfügt. Das Programm verfügt über alle gewünschten Funktionen, die vorher definiert wurden und bietet die Möglichkeit alle Szenarios zu bearbeiten. Dies wurde durch Verifikation mit diesen Szenarios überprüft und weitere Verbesserungsmöglichkeiten für die Zukunft aufgezeigt.