\chapter{Verbesserungsmöglichkeiten und Fazit}

\section{Verbesserungsmöglichkeitenn}

Eine Skalierung ist möglich, allerdings ist zu berücksichtigen, dass jede neue Sportart sowohl in der Ontologie, als auch in der Datenbank definiert werden muss mitsamt allen Informationen dazu. Dies kann sehr arbeitsintensiv werden. Bezüglich der Software bietet sich an, eine entsprechende Erweiterung zu entwickeln, die neue Sportarten erfassen kann. 

Die Datenbank liegt auf einem Server im Internet. In der Realität wäre eine Stand-Alone-Anwendung denkbar. In diesem Fall stellt sich die Frage, ob eine andere Form der Datenhaltung als eine MySQL-Datenbank hier zweckmäßiger wäre. Auf webbasierte Anwendungen ist das Backend ausgelegt. Da die Zugangsdaten hardcodiert sind, ist auf Datenbankseite hier auch darauf zu achten, dass entsprechende Sicherheitsvorkehrungen getroffen werden (z. B. kein Schreibzugriff auf die Datenbank).

Die Zeitangabe in einem Unterfenster ist nicht optimal, da bei Angaben einzelner Stunden am Tag für jede Stunde ein Klick gemacht werden muss. Dies wäre durch zielgerichteteres Eventhanling oder eine Andere Zeiterfassung möglich.

\section{Fazit}

Die Lösung hat zu einem vorzeigbaren Ergebnis geführt, das von der Bedienbarkeit sehr intuitiv ist. Von der Funktionalität her tut das Programm, was es soll. Die geforderte Funktionalität wurde erfüllt. Die wurde durch Verifikation mit den Szenarios gezeigt.